%!TEX TS-program = xelatex
%!TEX encoding = UTF-8 Unicode
\documentclass[11pt, a4paper]{./Awesome-CV/awesome-cv}
\geometry{left=3.1cm, top=1.8cm, right=3.1cm, bottom=1.8cm, footskip=.5cm}% Configure page margins with geometry
\usepackage{wallpaper}
% Awesome CV LaTeX Template for CV/Resume
%
% This template has been downloaded from:
% https://github.com/posquit0/Awesome-CV
%----------------------------------------------------------
% CONFIGURATIONS
%----------------------------------------------------------
\fontdir[fonts/]
% Color for highlights
\colorlet{awesome}{awesome-red}
% Awesome Colors: awesome-emerald, awesome-skyblue, awesome-red, awesome-pink, awesome-orange awesome-nephritis, awesome-concrete, awesome-darknight
\setbool{acvSectionColorHighlight}{true}
% Set false if you don't want to highlight section with awesome color
% Uncomment if you would like to specify your own color
% \definecolor{awesome}{HTML}{CA63A8}
% Colors for text
% Uncomment if you would like to specify your own color
% \definecolor{darktext}{HTML}{414141}
% \definecolor{text}{HTML}{333333}
% \definecolor{graytext}{HTML}{5D5D5D}
% \definecolor{lighttext}{HTML}{999999}

\renewcommand{\acvHeaderSocialSep}{\quad\textbar\quad} %info seperator
% If you would like to change the social information separator from a pipe (|) to something else
%	PERSONAL INFORMATION
%--------------------------------------------------------------------------
% Available options: circle|rectangle,edge/noedge,left/right
% \photo[rectangle,edge,right]{./examples/profile}
\name{Siddharth}{Reed}
\address{1088 Sunset Dr., Kelowna, BC, Canada, V1Y 9W1}
\mobile{647 822 9846}
%%% Social
\email{slreed@andrew.cmu.edu}
%\homepage{www.posquit0.com}
\github{DJSiddharthVader} %can change link in awesome-cv.cls
\linkedin{Sid-Reed} %can change link in awesome-cv.cls
%%% Optionals
%\position{Bioinformatics Intern}%{\enskip\cdotp\enskip}Security Expert}
% \gitlab{gitlab-id}
% \stackoverflow{SO-id}{SO-name}
% \twitter{@twit}
% \skype{skype-id}
% \reddit{reddit-id}
% \extrainfo{extra informations}
%\quote{``Be the change that you want to see in the world."}



%               LETTER INFORMATION
%---------------------------------------------------------
%\position{Bioinformatics Summer Intern}
\letterdate{\today}
%\lettertitle{Application for Bioinformatics Summer Intern}
\recipient{GRAIL} %company name
          {1525 O'Brien Drive\\
           Menlo Park, CA 94025}
\letteropening{Dear Hiring Manager,}% How the letter is opened
\letterclosing{Thank you for your time and consideration.\\ Sincerely,}
%\letterenclosure[Attached]{Résumé}% Any enclosures with the letter
\ULCornerWallPaper{1}{CMULetterHead.pdf}
\begin{document}
\makecvheader[-0.22in]{R}{16pt} %Give optional alignment (C: center, L: left, R: right)
\topskip0pt
\vspace*{\fill}
%\vspace{0.2in}
\makelettertitle %Print the title with above letter information

\begin{cvletter}
\hspace{8mm} I am an first year MSc. student in Computational Biology at Carnegie Mellon University applying for the bioinformatics summer internship.
\par \hspace{8mm}
Cancer is to vast and complex a beast to try and detect, let alone detect early and robustly, that you would certainly need equally vast and complex data to study it.
Of course this is a lesson I had to learn several times over throughout my career, that things in biology are rarely simple.
Shortly after I discovered bioinformatics as a field in my first year I spend what time and electives I had shoring up my quantitative skills, mostly independently.
I did take an excellent course on software design that introduced me to concepts like terminal commands, version control with git and some object oriented programming with python.
I was also fortunate to have an advisor who preached to me the gospel of Linux tome.
After finding out how powerful these tools could be I kept trying to improve myself, using them for everything.
\par \hspace{8mm}
Learning to code and analyze data are interesting enough on their own, but biological data presents many unique challenges among analytical fields.
The desire to try and discern meaningful, clear patterns from such messy subjects helped lead me to work studying cancers.
I conducted an internship in the Michael Hoffman lab at the University of Toronto under the supervision of Post-Doc and placenta expert Samantha Wilson.
She conceptualized the project of comparing transcriptomes profiles of various cancers and the placenta, trying to find what the placenta was regulating “correctly”that cancers cells were not.
I worked with the recount2 data and ended up reading a lot about how transcriptomic data is produced to ensure I knew what to expect and how to work with it.
I also spent much of my time visualizing my results and presenting them to experts and non-experts where the placenta and cancers differ and what that could imply for future cancer research.
\par \hspace{8mm}
I know my educational background in genetics, experience characterizing cancers and enthusiasm for intricate biological problems makes me especially suited for this internship.

\end{cvletter}
\makeletterclosing
\vspace*{\fill}
\makecvfooter{Siddharth Reed}{}{Cover Letter} %Print the footer with 3 arguments(<left>, <center>, <right>)}
\end{document}


%!TEX TS-program = xelatex
%!TEX encoding = UTF-8 Unicode
\documentclass[11pt, a4paper]{./Awesome-CV/awesome-cv}
\geometry{left=2.5cm, top=1.5cm, right=2cm, bottom=1.5cm, footskip=.5cm}% Configure page margins with geometry
\usepackage{wallpaper}
% Awesome CV LaTeX Template for CV/Resume
%
% This template has been downloaded from:
% https://github.com/posquit0/Awesome-CV
%----------------------------------------------------------
% CONFIGURATIONS
%----------------------------------------------------------
\fontdir[fonts/]
% Color for highlights
\colorlet{awesome}{awesome-red}
% Awesome Colors: awesome-emerald, awesome-skyblue, awesome-red, awesome-pink, awesome-orange awesome-nephritis, awesome-concrete, awesome-darknight
\setbool{acvSectionColorHighlight}{true}
% Set false if you don't want to highlight section with awesome color
% Uncomment if you would like to specify your own color
% \definecolor{awesome}{HTML}{CA63A8}
% Colors for text
% Uncomment if you would like to specify your own color
% \definecolor{darktext}{HTML}{414141}
% \definecolor{text}{HTML}{333333}
% \definecolor{graytext}{HTML}{5D5D5D}
% \definecolor{lighttext}{HTML}{999999}

\renewcommand{\acvHeaderSocialSep}{\quad\textbar\quad} %info seperator
% If you would like to change the social information separator from a pipe (|) to something else
%	PERSONAL INFORMATION
%--------------------------------------------------------------------------
% Available options: circle|rectangle,edge/noedge,left/right
% \photo[rectangle,edge,right]{./examples/profile}
\name{Siddharth}{Reed}
\address{1088 Sunset Dr., Kelowna, BC, Canada, V1Y 9W1}
\mobile{647 822 9846}
%%% Social
\email{slreed@andrew.cmu.edu}
%\homepage{www.posquit0.com}
\github{DJSiddharthVader} %can change link in awesome-cv.cls
\linkedin{Sid-Reed} %can change link in awesome-cv.cls
%%% Optionals
%\position{Bioinformatics Intern}%{\enskip\cdotp\enskip}Security Expert}
% \gitlab{gitlab-id}
% \stackoverflow{SO-id}{SO-name}
% \twitter{@twit}
% \skype{skype-id}
% \reddit{reddit-id}
% \extrainfo{extra informations}
%\quote{``Be the change that you want to see in the world."}



%               LETTER INFORMATION
%---------------------------------------------------------
% \vspace{-0.2in}
\ULCornerWallPaper{0.8}{CMULetterHead.pdf}
%\position{Bioinformatics Summer Intern}
\letterdate{\today}
\lettertitle{Application for Bioinformatics Systems Analyst}
\recipient{Avillach Lab} %company name
          {Department of Biomedical Informatics \\
           Harvard University \\
           10 Shattuck Street, Suite 514 \\
           Boston, MA 02115}
\letteropening{To whom it may concern,}% How the letter is opened
\letterclosing{Thank you for your time and consideration.\\ Sincerely,}
%\letterenclosure[Attached]{Résumé}% Any enclosures with the letter
\begin{document}
\makecvheader[-0.22in]{R}{16pt}{0.05in}
\topskip0pt
% \vspace*{\fill}
%\vspace{0.2in}
\makelettertitle %Print the title with above letter information

\begin{cvletter}

    I am graduating MSc. student in Computational Biology at Carnegie Mellon University
applying for the Bioinformatics Systems Analyst position.
% Undergrad
I think my diverse background and skills make me a great fit for this position.
My undergraduate degree taught me much about the complexities of biological systems and how we can try analyze them through experimentation.
At the same time, through my elective classes, internships and extracurricular projects I tried to develop my quantitative skills, mainly programming and analyzing data.

My time working for the start-up Adapsyn introduced me to a fast-paced environment, working with large datasets to help guide biochemists on what to investigate and learning about how to balance what was biologically informative and computationally feasible.
Later during my thesis and my research internship I was more involved with developing larger-scale, more comprehensive analyses for research problems.
In both cases I spent months curating data, cleaning/processing it, developing methods analysis and presenting my results to my supervisors.
This regular cycle of discovery i.e. picking a strategy, implementing it, dissecting the results and coming up with new questions really made me realize that I love bioinformatics, especially integrating various kinds of data (ontologies, phylogenies, various -omics etc.) to try to find interesting connections that can teach me about the underlying biology.

% MSc Experience
During my MSc. I have been improving my knowledge of bioinformatic and statistical methods and applying these through research and course projects.
Through the practice of writing more code and working as a TA I have developed a strong appreciation and practice of writing clean, readable code and comprehensive documentation.
Through courses I learned about the methods and implementations behind various genomic analyses methods, developing pipelines, simulating bacterial populations among other problems.
All of this has reinforced my love of poking at data and developing tools to do the poking, especially if those tools can help other researchers.
% So in my career I have lots of experience with both the practical issues of collecting with data, implementing workflows, working at scale and the conceptual issues of developing questions, designing experiments/analyses, communicating results.

% Good Fit
Making large, diverse biomedical datasets accessible to researchers is key for making new discoveries and validating known results.
In order to do so a  platform like needs to be easy to understand, easy to use and have plenty of documentation and examples to show how and why we as bioinformaticians do things.
It has even been shown in \clink{https://www.ncbi.nlm.nih.gov/pmc/articles/PMC6054259/}{the literature} that detailed usage instructions and examples are key to getting people to actually use your tools and always wanted by users, something I am quite personally sympathetic to.
I feel that my priorities and skills as a researcher align well with a platform like PIC-SURE, making data and analysis methods accessible to researchers while keeping a focus on rigor, transparency and reproducibility. 
I would love the opportunity to maintain and develop a platform like PIC-SURE and I know I would be a great fit for this position.

\end{cvletter}
\makeletterclosing
% \vspace*{\fill}
\makecvfooter{Siddharth Reed}{}{Cover Letter} %Print the footer with 3 arguments(<left>, <center>, <right>)}
\end{document}

